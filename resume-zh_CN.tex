% !TEX program = xelatex

\documentclass{resume}
\usepackage{zh_CN-Adobefonts_external} % Simplified Chinese Support using external fonts (./fonts/zh_CN-Adobe/)
%\usepackage{zh_CN-Adobefonts_internal} % Simplified Chinese Support using system fonts

\begin{document}
\pagenumbering{gobble} % suppress displaying page number

\name{Yifan Wu}

\basicInfo{
  \email{yifanwu@pku.edu.cn} \textperiodcentered\ 
  \phone{(+86) 184-8207-1800} \textperiodcentered\ 
  \github[wuyifan18]{https://github.com/wuyifan18}}

\section{教育经历}
\datedsubsection{\textbf{Peking University}}{Sep. 2019 -- Present}
Ph.D. student in Software Engineering\\
Advisor: Prof. \href{http://www.ss.pku.edu.cn/index.php/teacherteam/teacherlist/1674-%E6%9D%8E%E5%BD%B1}{Ying Li}
\datedsubsection{\textbf{University of Electronic Science and Technology of China}}{Sep. 2015 -- Jun. 2019}
B.E. in Software Engineering\\
GPA: 3.86/4.0\\
Rank: 3/132

\section{研究兴趣}
Graph Computing, Machine Learning Systems, Distributed Systems

\section{项目经历}
\datedsubsection{\textbf{LogFlash}}{Sep. 2019 -- Present}
\textit{Submitted} to anonymous peer-review, Second author\\
It is important to detect and diagnose anomalies accurately and timely for large-scale software systems. Thus, We propose LogFlash, an online self-updating anomaly detection and diagnosis approach that enables model training and anomaly diagnosis at real time. LogFlash considers anomaly detection and diagnosis as a real-time streaming processing task where each log entry is processed only once without any iterations or intermediate storage. Experiment results show that LogFlash reduces over 5 times of training and detection time compared with the state-of-art works while maintaining the capability of accurate problem diagnosis.
\begin{itemize}
  \item Implementation of log template mining module based on Flink
  \item Implementation of front-end page including task management and config management
  \item Deployment of LogFlash based on docker
\end{itemize}

\datedsubsection{\textbf{Data annotation platform}}{Apr. 2019 -- Aug. 2019}
\role{SDE Intern at Infimind}{}
Data annotation platform composed of multiple modules such as user login, project and task management, annotation statistics and review, annotation operation.
\begin{itemize}
  \item Implementation of front-end page including user based on Vue.js
\end{itemize}

\datedsubsection{\textbf{Crawler project}}{Mar. 2019 -- Apr. 2019}
\role{SDE Intern at Infimind}{}
An crawler tool for China Judgements Online.
\begin{itemize}
  \item Support IP proxy
  \item Support multiple processes
  \item Support full crawling
  \item Divide data according to decision time, region and court
\end{itemize}

\datedsubsection{\textbf{Xlearn Project}}{Feb. 2018 -- May. 2018}
\role{SDE Intern at School of Software, Tsinghua University}{Supvised by Mingsheng Long, Zhongyi Pei}
An platform where non-machine learning experts can run machine learning applications such as radar extrapolation.
\begin{itemize}
  \item Implementation of backend based on Flask and Nginx
  \item Deployment of algorithms such as radar extrapolation
  \item Implementation of front-end page
\end{itemize}

\section{主要论文}
\begin{enumerate}[parsep=0.5ex]
  \item How Far Have We Come in Detecting Anomalies in Distributed Systems? An Empirical Study with a Statement-level Fault Injection Method\\
  Yong Yang, \textbf{Yifan Wu}, Karthik Pattabiraman, Long Wang, Ying Li\\
  The 31st International Symposium on Software Reliability Engineering (\textbf{ISSRE'20}) (CCF B)
\item LogFlash: Online Anomaly Detection and Diagnosis from System Logs for Large-scale Software Systems (Submitted to anonymous peer-review)\\
  Tong Jia, Ying Li, \textbf{Yifan Wu}\\
\end{enumerate}

\section{主要奖励}
\begin{itemize}[parsep=0.5ex]
  \item \datedline{Outstanding Graduates of UESTC}{2019}
  \item \datedline{Second Award, People's Scholarship}{2018}
  \item \datedline{Suzhou Industrial Park Scholarship}{2017}
  \item \datedline{Outstanding League Member of UESTC}{2017}
  \item \datedline{First Award, People's Scholarship}{2016}
\end{itemize}

\section{专业技能}
\begin{itemize}[parsep=0.5ex]
  \item Languages: Java, Python, C, Go, Markdown
  \item Systems: Flink, Pytorch, Docker, TensorFlow, PowerGraph, Vue.js
\end{itemize}

%% Reference
%\newpage
%\bibliographystyle{IEEETran}
%\bibliography{mycite}
\end{document}
